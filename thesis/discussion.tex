\documentclass{article}
\bibliographystyle{unsrt}
\usepackage[dvipdfmx]{graphicx}
\usepackage[dvipdfmx,svgnames]{xcolor}%before tikz
\usepackage{tikz}
\usepackage{geometry}
\usepackage{amsmath,amssymb}
\usepackage{chemarrow}
\usepackage{booktabs}
\usepackage{longtable}
\usepackage{colortbl}
\usepackage{tabularx}
\usepackage{tabu}
\usepackage{float}
\usepackage{txfonts}
\usepackage{url}
\usepackage{siunitx}
\usepackage{subfigure}
\usepackage{enumerate}
\usepackage{caption}
\usepackage{wrapfig}
\usepackage{array}
\usepackage{titlesec}
\usepackage{picture}
\usepackage{multicol}
\usepackage{etoolbox}
\usepackage{fancyhdr}
\usepackage[]{algorithm2e}
\usepackage{algorithmic}
\usepackage{mathtools}
% \usepackage{listings,jlisting}
% \lstset{basicstyle=\ttfamily,
%   showstringspaces=false,
%   commentstyle=\color{red},
%   keywordstyle=\color{blue}
% }
%\usepackage{makeidx}

\newcommand{\napier}{\mathrm{e}}
\newcommand{\unitone}[2]{${#1}\,\mathrm{#2}$}
\newcommand{\unittwo}[3]{${#1}\,\mathrm{#2}/\mathrm{#3}$}
\newcommand{\dif}{\mathrm{d}}
\newcommand{\bibun}[2]{\frac{d {#1}}{d {#2}}}
\newcommand{\henbibun}[2]{\frac{\partial {#1}}{\partial {#2}}}
\newcommand{\figref}[1]{Figure \ref{#1}}
\newcommand{\tabref}[1]{Table \ref{#1}}
\newcommand{\equationref}[1]{equation \eqref{#1}}
\newcommand{\argmax}{\mathop{\rm arg~max}\limits}
\newcommand{\argmin}{\mathop{\rm arg~min}\limits}
\newcommand{\numberthis}{\addtocounter{equation}{1}\tag{\theequation}}
\renewcommand{\thesubfigure}{(\thefigure.\arabic{subfigure})}
\geometry{left=30mm, right=30mm, top=20mm, bottom=30mm}
\mathtoolsset{showonlyrefs}
%\makeindex


\begin{document}

\section{Discussion}
For the precise birth time parameter estimation, thousands of SNV loci are required, which is equivalent to the number of SNVs identified in the whole-genome sequencing. From the abundance ratio and inferred birth time of each subtype, we can estimate the growth rate of each subtype assuming exponential growth. Given one time point, we can only estimate the relative growth rate of the subtype compared to one another. In the future work, however, given sequential time point data, we would predict how each subtype will proliferate in the future. It is a notable advantage of our method which would be useful in the future cancer prognosis and treatment.

\end{document}
