\documentclass{article}
\bibliographystyle{unsrt}
\usepackage[dvipdfmx]{graphicx}
\usepackage[dvipdfmx,svgnames]{xcolor}%before tikz
\usepackage{tikz}
\usepackage{geometry}
\usepackage{amsmath,amssymb}
\usepackage{chemarrow}
\usepackage{booktabs}
\usepackage{longtable}
\usepackage{colortbl}
\usepackage{tabularx}
\usepackage{tabu}
\usepackage{float}
\usepackage{txfonts}
\usepackage{url}
\usepackage{siunitx}
\usepackage{subfigure}
\usepackage{enumerate}
\usepackage{caption}
\usepackage{wrapfig}
\usepackage{array}
\usepackage{titlesec}
\usepackage{picture}
\usepackage{multicol}
\usepackage{etoolbox}
\usepackage{fancyhdr}
\usepackage[]{algorithm2e}
\usepackage{algorithmic}
\usepackage{mathtools}
% \usepackage{listings,jlisting}
% \lstset{basicstyle=\ttfamily,
%   showstringspaces=false,
%   commentstyle=\color{red},
%   keywordstyle=\color{blue}
% }
%\usepackage{makeidx}

\newcommand{\napier}{\mathrm{e}}
\newcommand{\unitone}[2]{${#1}\,\mathrm{#2}$}
\newcommand{\unittwo}[3]{${#1}\,\mathrm{#2}/\mathrm{#3}$}
\newcommand{\dif}{\mathrm{d}}
\newcommand{\bibun}[2]{\frac{d {#1}}{d {#2}}}
\newcommand{\henbibun}[2]{\frac{\partial {#1}}{\partial {#2}}}
\newcommand{\figref}[1]{Figure \ref{#1}}
\newcommand{\tabref}[1]{Table \ref{#1}}
\newcommand{\equationref}[1]{equation \eqref{#1}}
\newcommand{\argmax}{\mathop{\rm arg~max}\limits}
\newcommand{\argmin}{\mathop{\rm arg~min}\limits}
\newcommand{\numberthis}{\addtocounter{equation}{1}\tag{\theequation}}
\renewcommand{\thesubfigure}{(\thefigure.\arabic{subfigure})}
\geometry{left=30mm, right=30mm, top=20mm, bottom=30mm}
\mathtoolsset{showonlyrefs}
%\makeindex


\begin{document}

\section*{Abstract}
Cancer arises as a result of the somatic mutation accumulation. Every cell within a tumor has derived from a single founder cell, whose subsequent accumulation of advantageous mutations causes clonal expansions (Figure 1). In the course of clonal expansion, a driver mutation gives rise to another type of clone, which is called a subtype. As a result, a tumor is a mixture of various subtypes. In the latest cancer treatment, it is important to identify what subtypes the tumor consists of and to identify the growth rates of these subtypes. For example, breast cancer subtypes have been studied well, and the clinical practice guidelines depending on each subtype have been established. Although well studied subtypes such as luminal A, luminal B, and HRE2+ could be identified biochemically, however, there is little as yet known about the growth rates of them. Furthermore, patient specific subtypes and unrevealed subtypes of other cancers cannot be identified with such chemical tests.

The emergence of the next-generation sequencers (NGSs) has enabled us to analyse whole cancer genomes at a single nucleotide resolution. Furthermore, using the latest single-cell sequencing technology, we can investigate the copy number and the genotype of each cell to identify whole subtypes in a tumor. However, sequencing a bulk tumor is still common because of technical difficulties and high cost of the single cell sequencing. Thus, our problem is to identify what kinds of subtypes the tumor consists of and to identify the characteristics of each subtype from NGS reads of the bulk tumor. To solve this problem, several methods such as PyClone~\cite{roth2014pyclone} and AncesTree~\cite{el2015reconstruction} have been proposed in previous works. However, we cannot estimate how rapidly each subtype proliferates and when these subtypes arose using these methods. Here we provide a statistical model to estimate birth time and growth rate of each subtype.

\end{document}
