\documentclass{article}
\bibliographystyle{unsrt}
\usepackage[dvipdfmx]{graphicx}
\usepackage[dvipdfmx,svgnames]{xcolor}%before tikz
\usepackage{tikz}
\usepackage{geometry}
\usepackage{amsmath,amssymb}
\usepackage{chemarrow}
\usepackage{booktabs}
\usepackage{longtable}
\usepackage{colortbl}
\usepackage{tabularx}
\usepackage{tabu}
\usepackage{float}
\usepackage{txfonts}
\usepackage{url}
\usepackage{siunitx}
\usepackage{subfigure}
\usepackage{enumerate}
\usepackage{caption}
\usepackage{wrapfig}
\usepackage{array}
\usepackage{titlesec}
\usepackage{picture}
\usepackage{multicol}
\usepackage{etoolbox}
\usepackage{fancyhdr}
\usepackage[]{algorithm2e}
\usepackage{algorithmic}
\usepackage{mathtools}
% \usepackage{listings,jlisting}
% \lstset{basicstyle=\ttfamily,
%   showstringspaces=false,
%   commentstyle=\color{red},
%   keywordstyle=\color{blue}
% }
%\usepackage{makeidx}

\newcommand{\napier}{\mathrm{e}}
\newcommand{\unitone}[2]{${#1}\,\mathrm{#2}$}
\newcommand{\unittwo}[3]{${#1}\,\mathrm{#2}/\mathrm{#3}$}
\newcommand{\dif}{\mathrm{d}}
\newcommand{\bibun}[2]{\frac{d {#1}}{d {#2}}}
\newcommand{\henbibun}[2]{\frac{\partial {#1}}{\partial {#2}}}
\newcommand{\figref}[1]{Figure \ref{#1}}
\newcommand{\tabref}[1]{Table \ref{#1}}
\newcommand{\equationref}[1]{equation \eqref{#1}}
\newcommand{\argmax}{\mathop{\rm arg~max}\limits}
\newcommand{\argmin}{\mathop{\rm arg~min}\limits}
\newcommand{\numberthis}{\addtocounter{equation}{1}\tag{\theequation}}
\renewcommand{\thesubfigure}{(\thefigure.\arabic{subfigure})}
\geometry{left=30mm, right=30mm, top=20mm, bottom=30mm}
\mathtoolsset{showonlyrefs}
%\makeindex


\begin{document}

\section*{Supplementary Figures}
\begin{figure}[H]
 \includegraphics[bb=0 0 707 551,width=0.45\hsize]{../../cnvkit/TB-scatter.jpg}
 \caption{$log(2)$ copy ratio between the tumor and the normal sample detected using copy number detection tool CNVkit~\cite{talevich2016cnvkit}}
\end{figure}

\begin{figure}[H]
  \includegraphics[bb=16 6 838 212,width=0.7\hsize]{../../THetA/TB.n2.graph.pdf}
 \caption{Abundance ratio estimation of the normal ant tumor subtypes using THetA}
\end{figure}

\begin{figure}[H]
 \subfigure[]{\includegraphics[bb=1 15 467 480,width=0.4\hsize]{../../mutect/mutect/TB_nextera_3/TB.call_stats.extract.tumor_f.sorted.pdf}}
% \subfigure[]{\includegraphics[bb=1 15 479 480,width=0.4\hsize]{../../mutect/mutect/TB_nextera_3/TB.call_stats.extract.keep.tumor_f.sorted.pdf}}
\end{figure}

\begin{figure}[H]
 \subfigure{\includegraphics[bb=0 0 700 700,width=0.3\hsize]{../../mutect/mutect/TB_nextera_3/TB.call_stats.extract.tumor_f.inverse.sorted.jpg}}
 \subfigure{\includegraphics[bb=0 0 700 700,width=0.3\hsize]{../../mutect/mutect/TB_nextera_3/TB.call_stats.extract.tumor_f.inverse.sorted.1-100.jpg}}
 \subfigure{\includegraphics[bb=0 0 700 700,width=0.3\hsize]{../../mutect/mutect/TB_nextera_3/TB.call_stats.extract.tumor_f.inverse.sorted.0.12-0.25.jpg}}
% \subfigure[]{\includegraphics[bb=1 15 481 480,width=0.45\hsize]{../../mutect/mutect/TB_nextera_3/TB.call_stats.extract.tumor_f.inverse.sorted.pdf}}
% \subfigure[]{\includegraphics[bb=1 15 475 480,width=0.3\hsize]{../../mutect/mutect/TB_nextera_3/TB.call_stats.extract.tumor_f.inverse.sorted.1-100.pdf}}
% \subfigure[]{\includegraphics[bb=1 15 475 480,width=0.3\hsize]{../../mutect/mutect/TB_nextera_3/TB.call_stats.extract.tumor_f.inverse.sorted.0.12-0.25.pdf}}
\end{figure}

\begin{figure}[H]
 \includegraphics[bb=1 15 467 481,width=0.3\hsize]{../src/read_generation/generated/subtype2_topology0_snv_forvaf/vaf_coverage100/0.5/0.4/3/coverage100_snv1000_seed1.reads.1.vaf.pdf}
 \includegraphics[bb=1 15 467 481,width=0.3\hsize]{../src/read_generation/generated/subtype2_topology0_snv_forvaf/vaf_coverage100/0.5/0.4/3/coverage100_snv1000_seed1.reads.2.vaf.pdf}
 \includegraphics[bb=1 15 467 481,width=0.3\hsize]{../src/read_generation/generated/subtype2_topology0_snv_forvaf/vaf_coverage100/0.5/0.4/3/coverage100_snv1000_seed1.reads.1.vaf_coverage100_snv1000_seed1.reads.2.vaf.pdf}
\end{figure}

\begin{figure}[H]
 \includegraphics[bb=3 28 487 471,width=0.3\hsize]{../src/read_generation/generated/subtype2_topology0_snv_forvaf/neutral_evolution_plot/0.5/0.4/3/coverage100_snv1000_seed1.reads.1.vaf.filtered_0.05_0.2.reverse.pdf}
 \includegraphics[bb=3 28 487 471,width=0.3\hsize]{../src/read_generation/generated/subtype2_topology0_snv_forvaf/neutral_evolution_plot/0.5/0.4/3/coverage100_snv1000_seed1.reads.2.vaf.filtered_0.1_0.5.reverse.pdf}
 \includegraphics[bb=3 28 487 471,width=0.3\hsize]{../src/read_generation/generated/subtype2_topology0_snv_forvaf/neutral_evolution_plot/0.5/0.4/3/coverage100_snv1000_seed1.reads.vaf.filtered_0.12_0.24_with_grad.reverse.pdf}
\end{figure}

\begin{figure}[H]
 \subfigure[Accuracy of the birth time ($t$) estimation]{\includegraphics[bb=3 15 475 446,width=0.45\hsize]{../src/alpha_n_subtype_pattern_multinomial/accuracy_result_seed_poisson/4_2_original_total1000000_td1000_cell1000000.rmsd.all.t_snv.pdf}}
 \subfigure[Accuracy of the abundance ratio ($n$) estimation]{\includegraphics[bb=3 15 475 446,width=0.45\hsize]{../src/alpha_n_subtype_pattern_multinomial/accuracy_result_seed_poisson/4_2_original_total1000000_td1000_cell1000000.rmsd.all.n_snv.pdf}}
   \caption{
 Accuracy of the birth time (t) and abundance ratio (n) estimation against the number of SNVs. Outliers are detected by means of deviation from the $1.5 \times IQR$, where $IQR$ is the interquartile range. Whiskers show the maximum and minimum estimate except for the outliers. If we do not know the subtypein which each SNV is originated, gradient descent estimations fall into the optima which are different from the true parameters. Thus, I could not estimate the true birth time and abundance ratio even if there are thousands of SNVs by marginalizing the subtypes with equal probabilities.
  }
 \label{fig: binom_subtype_ungiven}
\end{figure}

\begin{figure}[H]
 \subfigure[Accuracy of the birth time ($t$) estimation]{\includegraphics[bb=3 15 475 453,width=0.45\hsize]{../src/alpha_n_subtype_pattern_multinomial_mle/accuracy_result_seed_poisson/4_2_original_total1000000_td1000_cell1000000.rmsd.all.t_snv.pdf}}
  \subfigure[Accuracy of the abundance ratio ($n$) estimation]{\includegraphics[bb=3 15 475 446,width=0.45\hsize]{../src/alpha_n_subtype_pattern_multinomial_mle/accuracy_result_seed_poisson/4_2_original_total1000000_td1000_cell1000000.rmsd.all.n_snv.pdf}}
   \caption{
 Accuracy of the birth time (t) and abundance ratio (n) estimation against the number of SNVs. Outliers are detected by means of deviation from the $1.5 \times IQR$, where $IQR$ is the interquartile range. Whiskers show the maximum and minimum estimate except for the outliers. If we do not know the subtypein which each SNV is originated, gradient descent estimations fall into the optima which are different from the true parameters. Thus, I could not estimate the true birth time and abundance ratio even if there are thousands of SNVs by adding only the maximum probability component with regard to the subtype in which each SNV is originated.
  }
 \label{fig: binom_subtype_ungiven_mle}
\end{figure}

\end{document}
